\section{Conclusiones y recomendaciones}
A modo de conclusión, es de gran de satisfacción haber realizado un entrenamiento adecuado y así obtener los resultados esperados, es decir, que el microcontrolador logre procesar todas las señales para abrir o cerrar una puerta. Aunque algunas veces el sistema se acciona por si solo, y esto puede ser por ciertas razones, por ejemplo, la cantidad de muestras entrenadas, se esperaba hacer un procesamiento de datos más grandes pero Edge Impulse le costaba cada vez más incluso en repetidas ocasiones el algoritmo detrás de esto excedía el tiempo esperado por lo que terminaba desechar toda la grabación. Por tanto, hay que tener una paciencia (muy pequeña) y poder ver su funcionamiento. Este tipo de proyectos es de carácter muy enriquecedor porque implica mucha investigación, pruebas y errores, los cuales son enseñanzas muy buenas que ayudaron a avanzar en el desarrollo del proyecto. Es increíble la cantidad de proyectos de TinyML que se pueden hacer a partir de lo visto en el curso y lo bueno todo esto que hay muchos recursos para lograrlo.\par

Como recomendaciones:
\begin{itemize}

    \item Comprobar que el funcionamiento de los pines es el correcto, no asumir que todos están buenos y que te lleve a un cambio de planes en el diseño.
    \item Elegir un proyecto que sea realizable y que este de acuerdo a sus conocimientos. Esto por el poco tiempo que se dispone en el tercer ciclo.
    \item Tener mucho cuidado con el manejo de los componentes en todo momento porque causar algún daño en media planificación es muy doloroso por cuestiones de tiempo de más a invertir en la búsqueda del mismo.
    \item  Disfrutar del proceso porque es una etapa de mucho aprendizaje cuando se emprenden nuevos conocimientos sin saber nada.
\end{itemize}