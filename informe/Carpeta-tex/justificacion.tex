\section{Justificación}
La razón por la que se decidió realizar este proyecto se basa en la necesidad de crear un prototipo de caja fuerte inteligente que pueda ser desbloqueada tanto por comandos de voz como manualmente a través de un keypad. Esta combinación de métodos de acceso aumenta la comodidad y la accesibilidad para una variedad de personas, esto incluye a personas que tienen sus manos ocupadas en dicho momento o personas con limitaciones físicas que se les dificulta el ingreso de una clave convencional.
Además, el proyecto ayuda ampliar nuestro conocimiento y experiencia en el manejo de componentes físicos, como el keypad, el servo y la pantalla LCD (aunque esta última fue simulada pero no implementada). El proceso de trabajar con estos periféricos no solo fortalece nuestra comprensión de la electrónica y la programación, sino que también nos prepara para abordar proyectos más complejos en el futuro, donde la integración de múltiples dispositivos será fundamental.
Por último, la implementación de machine learning en un MCU es un aprendizaje de gran valor para futuros proyectos porque permite explorar aplicaciones que van más allá de las soluciones convencionales, abriendo nuevas posibilidades en el diseño y la implementación de sistemas inteligentes y adaptativos. 